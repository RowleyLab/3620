\documentclass[12pt, letterpaper]{article}
\usepackage{SyllabusStyle}

\begin{document}
\begin{center}
{\Large \textsc{Physical Chemistry II}}

CHEM 3620
\end{center}
\begin{center}
{\large Spring 2021}
\end{center}
\begin{center}
\rule{0.85\textwidth}{0.4pt}
\begin{tabular}{llcll}
\textbf{Instructor:} & Matthew Rowley & & \textbf{Office Hours:} & Daily 1:00 pm - 2:00 pm \\
\textbf{Telephone:} & (435) 586-7875 & & \textbf{Office:} & SC-220 \\
\textbf{Email:} & \multicolumn{3}{l}{matthewrowley$1$@suu.edu}\\
\multicolumn{5}{c}{Please include the course number in the subject line of all correspondence.} 
\end{tabular}
\rule{0.85\textwidth}{0.4pt}
\end{center}

\section*{Course Description} 
This course will provide an introduction to quantum mechanics, a topic that is the foundation of all chemistry. During the turn of the \nth{20} century many scientists believed that all of the major questions about science had already been answered and all that was needed was a little refinement. However, led by the experimental discoveries and theoretical models of Einstein, Planck, de Broglie, and others, the scientific world was turned upside down and the miniature physical world of quantum mechanics was discovered. This course will focus on quantum mechanics; full understanding of this topic will require proficient math skills. Therefore, it is highly recommended that students review the math chapters in the text.

\paragraph{Prerequisites:}
A minimum grade of C (2.0 or above) in CHEM 1220/1225 and Math 1220, and admission to the Chemistry program.

\paragraph{Concurrent requisite:}
CHEM 3625 -- Physical Chemistry Lab II

\paragraph{Course Materials:} ~

$\circ$ \emph{Physical Chemistry: Thermodynamics, Structure, and Change} by Atkins, de Paula, and Keeler -- 11th edition (required) ISBN: 978-0-19-876986-6

$\circ$ \emph{Applied Mathematics for Physical Chemistry} by James R. Barrante (suggested)

\vspace{-4pt}
---or---

\vspace{-4pt}
$\circ$ \emph{Mathematics for Physical Chemistry} by Robert G. Mortimer (suggested)

\paragraph{Student Learning Outcomes:}
\begin{itemize}
  \item{Students will successfully identify and understand the experiments that led to the discovery of quantum mechanics.}
  \item{Students will understand and utilize the fundamental quantum mechanical Schrödinger equation.}
  \item{Students will generate and apply the following mathematical models to real systems:}
  \begin{itemize}
    \item{Particle in a Box}
    \item{Harmonic Oscillator}
    \item{Rigid Rotor}
  \end{itemize}
  \item{Students will understand how the quantum mechanical models can be used in spectroscopy to predict various structural and chemical properties.}
  \item{Students will learn about group theory and how combining it with quantum mechanics provides a powerful tool for understanding the consequences of molecular symmetry (atomic orbital contributions to molecular orbitals, the origin of selection rules, identify the normal modes of vibration for molecules, and determine IR/Raman activity.}
\end{itemize}

\section*{Tentative Schedule}
This class will meet on Mondays, Wednesdays, and Fridays from 12:00 pm to 12:50 pm in room 226 of the Science Center (SC).

\noindent For the best lecture experience, read the indicated textbook chapter \emph{before} each lecture. 

\noindent
\begin{tabular}{rcccc}
	& Date && Topic & Chapter\\
	\midrule
	Week 1 & M, Jan. 11&& The Origins of Quantum Mechanics & 7A\\
	& W, Jan. 13&& Wavefunctions & 7B\\
	& F, Jan. 15&& Operators and Observables & 7C\\
	\midrule
	Week 2 & M, Jan. 18& \multicolumn{3}{l}{\textbf{Martin Luther King Day -- No Class!}}\\
	& W, Jan. 20&& Experiments and Interpretations in QM & **\\
	& F, Jan. 22&& Translational Motion & 7D\\
	\midrule
	Week 3 & M, Jan. 25&& Vibrational Motion & 7E\\
	& W, Jan. 27&& Rotational Motion & 7F\\
	& F, Jan. 29&& Hydrogenic Atoms & 8A\\
	\midrule
	Week 4 & M, Feb. 1&& Many Electron Atoms & 8B\\
	& W, Feb. 3&& Atomic Spectra & 8C\\
	& F, Feb. 5&& Valence-Bond Theory & 9A\\
\end{tabular}

\noindent
\begin{tabular}{rcccc}
	& Date && Topic & Chapter\\
	\midrule
	Week 5 & M, Feb. 8&& Molecular Orbital Theory: the Hydrogen Molecule-Ion & 9B\\
	& W, Feb. 10&& Molecular Orbital Theory: Homonuclear Diatomic Molecules & 9C\\
	& F, Feb. 12&& Molecular Orbital Theory: Heteronuclear Diatomic Molecules & 9D\\
	\midrule
	Week 6 & M, Feb. 15& \multicolumn{3}{l}{\textbf{President's Day -- No Class!}}\\
	& W, Feb. 17&& Molecular Orbital Theory: Polyatomic Molecules & 9E\\
	& F, Feb. 19&& Shape and Symmetry & 10A\\
	\midrule
	Week 7 & M, Feb. 22&& Group Theory & 10B\\
	& W, Feb. 24&& Applications of Symmetry & 10C\\
	& F, Feb. 26&& General Features of Molecular Spectroscopy & 11A\\
	\midrule
	Week 8 & M, Mar. 1& \multicolumn{3}{l}{\textbf{Spring Break -- No Class!}}\\
	& W, Mar. 3& \multicolumn{3}{l}{\textbf{Spring Break -- No Class!}}\\
	& F, Mar. 5& \multicolumn{3}{l}{\textbf{Spring Break -- No Class!}}\\
	\midrule
	Week 9 & M, Mar. 8&& Rotational Spectroscopy & 11B\\
	& W, Mar. 10&& Vibrational Spectroscopy of Diatomic Molecules & 11C\\
	& F, Mar. 12&& Vibrational Spectroscopy of Polyatomic Molecules & 11D\\
	\midrule
	Week 10 & M, Mar. 15&& Electronic Spectra & 11F\\
	& W, Mar. 17&& Decay of Excited States & 11G\\
	& F, Mar. 19&& Lasers and Spectroscopy Special Topics & **\\
	\midrule
	Week 11 & M, Mar. 22&& General Principles of NMR & 12A\\
	& W, Mar. 24&& Features of NMR Spectra & 12B\\
	& F, Mar. 26&& Pulse Technique in NMR & 12C\\
	\midrule
	Week 12 & M, Mar. 29&& Electron Paramagnetic Resonance & 12D\\
	& W, Mar. 31& \multicolumn{3}{l}{\textbf{Festival of Excellence -- No Class!}}\\
	& F, Apr. 2&& The Boltzmann Distribution & 13A\\
\end{tabular}

\noindent
\begin{tabular}{rcccc}
	& Date && Topic & Chapter\\
	\midrule
	Week 13 & M, Apr. 5&& Molecular Partition Functions & 13B\\
	& W, Apr. 7&& Molecular Energies & 13C\\
	& F, Apr. 9&& The Canonical Ensemble & 13D\\
	\midrule
	Week 14 & M, Apr. 12&& Internal Energy and Entropy & 13E\\
	& W, Apr. 14&& Derived Functions & 13F\\
	& F, Apr. 16&& Electric Properties of Molecules & 14A\\
	\midrule
	Week 15 & M, Apr. 19&& Interactions Between Molecules & 14B\\
	& W, Apr. 21&& Liquids & 14C\\
	& F, Apr. 23&& Macromolecules and Self-Assembly & 14D-E\\
	\midrule
	Finals Week& & \multicolumn{3}{l}{\textbf{Final Exam --- Details will be given closer to finals week}}\\
\end{tabular}

\section*{Course Policies and Grading}
Grades will be based on the following items:
\begin{description}
  \item[4 Midterm Exams] 40\%
  \item[Final Exam] 15\%
  \item[Quizzes] 15\%
  \item[Homework] 30\%
\end{description}
Final Grades will be assigned according to the following grade scale:

\begin{tabular}{cl|c|cl}
	Percentage & Grade &  & Percentage & Grade \\ \midrule
	100--93.0 & A     &  &  77.0--73.0 & C     \\
	93.0--90.0 & A-    &  &  73.0--70.0 & C-    \\
	90.0--87.0 & B+    &  &  70.0--67.0 & D+    \\
	87.0--83.0 & B     &  &  67.0--63.0 & D     \\
	83.0--80.0 & B-    &  &  63.0--60.0 & D-    \\
	80.0--77.0 & C+    &  &     < 60.0 & F
\end{tabular}
\paragraph{Midterm Exams:}
There are four midterm exams, to be completed in the testing center unless prior arrangements have been made. It is departmental policy that exams not be returned, although students may examine their completed exam and the answer key in my office. Scantron sheets are not required for midterms.

\paragraph{Final Exam:}
The final exam is comprehensive. The final is produced by the American Chemical Society, and the instructor will not have access to the exam prior to its administration. Therefore, it is to your advantage to learn as much as possible throughout the semester. The test is multiple choice and a scantron will be required.

\paragraph{Homework:}
The homework assignments will be a combination of problems from the textbook as well as problems that I have written or gathered from various other resources. These assignments are designed to be somewhat challenging and will require diligence in order to complete. They sometimes push the students to think \emph{beyond} the lecture material.

\paragraph{Quizzes:}
Quizzes will be shorter and more frequent assignments than the regular homework. The difficulty and scope of the questions on quizzes will more closely match what you will see on the exams.

\paragraph{Attendance Policy:}
Students are expected to attend class. If you must miss class, contact the instructor.

\paragraph{Make-up Work Policy:}
In general, there will be no opportunity to make up missed work. If you must miss class, please do any assigned work in advance, and arrange to turn it in early.

\section*{Miscellany}
\paragraph{Important syllabus statements related to ATTENDANCE and COVID-19:}
\begin{description}
	\item[Q:~~~~~~] If a student does NOT want to attend face-to-face this fall, is there another option?
	\item[A:~~~~~~] All students who would normally attend face to face classes should plan to do so, unless they are ill or are concerned for their health. However, if a student is ill or concerned for their health, students will be able to complete classes this fall whether they stay home or are here in Cedar City. Digital recording equipment will be installed in each instructional space (classrooms, labs and other venues) so that students can log in and attend face-to-face classes remotely. This will allow any SUU community member to engage in classroom activities in the way they feel comfortable. Faculty will also be able to teach remotely in accordance with their personal health needs. Faculty will likely post the link for how to listen/watch live in the syllabus (and/or Canvas) or simply email it to their students. We recommend to students that they reach out to their faculty to notify them how they will be 'attending' class.
	\item[Q:~~~~~~] Would it be possible to participate in courses remotely for the entire fall 2020 semester if desired?
	\item[A:~~~~~~] Yes. The intent is that this semester is normal and that students attend face to face classes, however, if they are symptomatic or concerned for their health, they can participate remotely. All students will be able to complete the fall semester from another location. Students should notify their faculty if they are completing the course from a distance rather than attending class in-person. International students will need to be mindful of how this will impact their non-immigrant status and to contact International Affairs if there are any questions.
	\item[Q:~~~~~~] Do students need special documentation to be allowed to attend classes remotely?
	\item[A:~~~~~~]  No, students do not need special documentation to attend classes remotely. However, students should notify their faculty if they are completing the course from a distance rather than attending class in-person so expectations can be clearly understood. Remote attendance should be for those who are ill or symptomatic.
\end{description}

\paragraph{ZOOM ETIQUETTE:}
Your class may utilize the Zoom online conference system. To participate in Zoom meetings, you will need to have a webcam/microphone or a smartphone with the Zoom app. We will adopt the same rules and norms as in a physical classroom (take notes; participate by asking and answering questions; wear classroom-ready clothing). For everyone’s benefit:
\begin{itemize}
	\item Join the course in a quiet, distraction free location
	\item Be aware of your background
	\item Turn on your video (you may close it after attendance is taken if your internet connection cannot handle having both audio and video going)
	\item Mute your microphone unless you are speaking
	\item Close browser tabs and software not required for participating in class
	\item Remember that our classes are in the Mountain Time zone
\end{itemize}
The success of this class will depend on the same commitment to learning we all typically bring to the physical classroom.

\paragraph{Scientific Calculator:}
There are many different ways to calculate figures during homework. It is tempting to rely on Online resources such as \href{http://www.wolframalpha.com}{http://www.wolframalpha.com}, or to simply use a calculator application on a smart phone. During exams, however, any devices capable of connecting to the Internet will \emph{not} be allowed. You will instead need a scientific calculator capable of performing exponentiation and logarithms for the exams. Using this calculator exclusively while doing homework will ensure that you are familiar with it for use during exams.

\paragraph{Academic Integrity:}
Scholastic dishonesty will not be tolerated and will be prosecuted to the fullest extent. You are expected to have read and understood the current issue of the \href{https://help.suu.edu/handbook}{Student Handbook} (published by Student Services) regarding student responsibilities and rights, and for the intellectual property policy, information about procedures, and what constitutes acceptable behavior. From University policy 6.33: ``The University defines plagiarism as intentionally or carelessly presenting the work of another as one’s own. It includes submitting an assignment purporting to be the student’s original work which has wholly or in part been created by another person, or cutting and pasting of source material\ldots''

\paragraph{ADA Policy:}
Students with medical, psychological, learning, or other disabilities desiring academic adjustments, accommodations, or auxiliary aids will need to contact the Southern Utah University Coordinator of Services for Students with Disabilities (SSD), in Room 206F of the Sharwan Smith Center or phone (435) 865-8022. SSD determines eligibility for and authorizes the provision of services.

\paragraph{Emergency Management Statement:}
In case of emergency, the university's Emergency Notification System (ENS) will be activated. Students are encouraged to maintain updated contact information using the link on the homepage of the \emph{mySUU} portal. In addition, students are encouraged to familiarize themselves with the Emergency Response Protocols posted in each classroom. Detailed information about the university's emergency management plan can be found at: \href{http://www.suu.edu/emergency}{http://www.suu.edu/emergency}

\paragraph{HEOA Compliance Statement:}
The sharing of copyrighted material through peer-to- peer (P2P) file sharing, except as provided under U.S. copyright law, is prohibited by law. Detailed information can be found at: \href{https://help.suu.edu/article/1097/p2p-and-copyright-infringement}{https://help.suu.edu/article/1097/p2p-and-copyright-infringement}

\paragraph{LINK Statement:}
SUU faculty and staff care about the success of our students. In addition to your professor, numerous services are available to assist you with the achievement of your educational goals. SUU's LINK system may be used by faculty to notify you and/or your advisors of their concern for your progress and provide references to campus services as appropriate.

\paragraph{SUUSA Statement:}
As a student at SUU, you have representation from the SUU Student Association (SUUSA) which advocates for student interests and helps work as a liaison between the students and the university administration. You can submit My SUU Voice feedback by going here: \href{https://www.suu.edu/suusa/voice}{https://www.suu.edu/suusa/voice} Likewise, you can learn more about SUUSA's Executive Council here (\href{https://www.suu.edu/suusa/executive-council/}{https://www.suu.edu/suusa/executive-council/}) and about indivdual SUUSA's Student Sentors here (\href{https://www.suu.edu/suusa/senate/}{https://www.suu.edu/suusa/senate/})

\paragraph{Disclaimer:}
Information contained in this syllabus, other than the grading, late assignments, make up work and attendance policies, may be subject to change as deemed appropriate by the instructor.
\end{document}