\documentclass[10pt, letterpaper]{memoir}
\usepackage{HomeworkStyle}

\begin{document}
\begin{center}
	{\large Homework 7.1 -- Introduction to Quantum Theory}
\end{center}

Name: \rule[-.1mm]{15em}{0.1pt}

\begin{description}
	\item [Excercise 7A.5 (b)] ~ (10 points)
	
	Calculate the energy of a photon and the energy per mole of photons for radiation of wavelength (i) $200~nm$ (ultraviolet), (ii) $150~pm$ (X-ray), (iii) $1.00~cm$ (microwave)
		
	\vspace{10em}
	\item [Excercise 7A.11 (a)] ~ (5 points)
	
	To what speed must an electron be accelerated for it to have a de Broglie wavelength of $3.0~cm$?
	
	\vspace{8em}
	\item [Exercise 7A.12 (a)] ~ (5 points)
	
	The `fine-structure constant,' $\alpha$, plays a special role in the structure of matter; its approximate value is $\nicefrac{1}{137}$. What is the de Broglie wavelength of an electron travelling at $\alpha c$, where $c$ is the speed of light?
	
	\vspace{10em}
	\item [Exercise 7B.2 (b)] ~ (5 points)
	
	Normalize (to $1$) the wavefunction $e^{-ax}$ in the range $0\leq x\leq\infty$, with $a>0$
	
	\vspace{15em}
	\item [Problem 7B.7] ~ (10 points)
	
	A normalized wavefunction for a particle confined between $0$ and $L$ in the x direction is $\Psi=(\nicefrac{2}{L})^{\nicefrac{1}{2}}\sin(\nicefrac{\pi x}{L})$. Suppose that $L = 10.0~nm$. Calculate the probability that the particle is (a) between $x=4.95~nm$ and $5.05~nm$, (b) between $x=1.95~nm$ and $2.05~nm$, (c) between $x=9.90~nm$ and $10.00~nm$, (d) between $x=5.00~nm$ and $10.00~nm$
	
	\vspace{15em}
	\item [Exercise 7C.2 (a)] ~ (10 points)
	
	Identify which of the following functions are eigenfunctions of the operator $\nicefrac{\mathrm{d}}{\mathrm{d}x}$: (i) $\cos(kx)$; (ii) $e^{ikx}$; (iii) $kx$, (iv) $e^{-ax^2}$. Give the corresponding eigenvalue where appropriate
	
	\vspace{10em}
	\item [Exercise 7C.3 (a)] ~ (5 points)
	
	Functions of the form $\sin(\nicefrac{n\pi x}{L})$, where $n=1,2,3,\ldots$ are wavefunctions in a region of length $L$ (between $x=0$ and $x=L$). Show that the wavefunctions with $n=1$ and $2$ are orthogonal; you will find the necessary integrals in the \emph{Resource section}. (\emph{Hint:} Recall that $\sin(n\pi)=0$ for integer $n$) (\emph{Even Better Hint:} Some integrals can be solved by symmetry arguments)
	
	\vspace{10em}
	\item [Exercise 7C.3 (b)] ~ (5 points)
	
	For the same system as in Exercise E7C.3(a) show that the wavefunctions with $n=2$ and $n=4$ are orthogonal
	
	\vspace{10em}
	\item [Exercise 7C.9 (a)] ~ (5 points)
	
	Calculate the minimum uncertainty in the speed of a ball of mass $500~g$ that is known to be within $10~\mu m$ of a certain point on a bat. What is the minimum uncertainty in the position of a bullet of mass $5.0~g$ that is known to have a speed somewhere between $350.000~01~ms^{-1}$ and $350.000~00~ms^{-1}$?
\end{description}
\end{document}
	