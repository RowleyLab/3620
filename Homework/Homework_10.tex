\documentclass[10pt, letterpaper]{memoir}
\usepackage{HomeworkStyle}

\begin{document}
\begin{center}
	{\large Homework 10 -- Molecular Symmetry}
\end{center}

Name: \rule[-.1mm]{15em}{0.1pt}

\begin{description}
	\item [Excercise 10A.4(a)] ~ (10 points)
	
	List the symmetry elements of the following molecules and name the point groups to which they belong: 
	\begin{description} 
		\item [(i)] \ch{NO2}  %  C2v
		
		\vspace{2em}
		\item [(ii)] \ch{PF5}  % D3h
		
		\vspace{2em}
		\item [(iii)] \ch{CHCl3}  % C3v
		
		\vspace{2em}
		\item [(iv)] 1,4-difluorobenzene  % D2h
	\end{description}
	
	\vspace{2em}
	\item [Exercise 10B.5(a)] ~ (10 points)
	
	By inspection of the character table for \ch{D_{3h}}, state the symmetry species of the $3p$ and $3d$ orbitals located on the central \ch{Al} atom in \ch{AlF3}  %linear and quadratic functions
	
	\vspace{10em}
	\item [Exercise 10B.7(a)] ~ (5 points)
	
	What is the maximum possible degree of degeneracy of the orbitals in benzene?  % 2 - from D6h character table representations
	
	\vspace{5em}
	\item [Exercise 10C.1(a)] ~ (5 points)
	
	Use symmetry properties to determine whether or not the integral $\int p_xzp_z\mathrm{d}\tau$ is necessarily zero in a molecule iwth symmetry $C_{2v}$
	
	\vspace{5em}
	\item [Exercise 10C.2(a)] ~ (5 points)
	
	Is the transition $A_1\rightarrow A_2$ forbidden for electric dipole transitions in a $C_{3v}$ molecule?
	
	\vspace{5em}
	\item [Exercise 10C.4(a)] ~ (10 points)
	
	Consider the $C_{2v}$ molecule \ch{OF2}; take the molecule to lie in the $yz$-plane, with $z$ directed along the $C_2$ axis; the mirror plane $\sigma^\prime_v$ is the $yz$-plane, and $\sigma_v$ is the $xz$-plane. The combination $p_z(A)+p_z(B)$ of the two F atoms spans $A_1$, and the combination $p_z(A)-p_z(B)$ of the two F atoms spans $B_2$. Are there any valence orbitals of the central O atom that can have a non-zero overlap with these combinations of F orbitals? How would the situation be different in \ch{SF2}, where $3d$ orbitals might be available?
	
	\vspace{20em}
	\item [Exercise 10C.6(a)] ~ (5 points)
	
	The ground state of \ch{NO2} is $A_1$ in the group $C_{2v}$. To what excited states may it be excited by electric dipole transitions, and what polarization of light is it necessary to use?
\end{description}
\end{document}
	