\documentclass[10pt, letterpaper]{memoir}
\usepackage{HomeworkStyle}

\begin{document}
\begin{center}
	{\large Homework 8 -- Atomic Structure and Spectra}
\end{center}

Name: \rule[-.1mm]{15em}{0.1pt}

\begin{description}
	\item [Excercise 8A.1(a)] ~ (5 points)
	
	State the orbital degeneracy of the levels in a hydrogen atom that have energy \\(i) $-hc\tilde{R}_H$ (ii) $-\frac{1}{9}hc\tilde{R}_H$ (iii) $-\frac{1}{25}hc\tilde{R}_H$
	
	\vspace{8em}
	\item [Exercise 8A.5(a)] ~ (5 points)
	
	At what radius does the probability density of an electron in the \ch{H} atom fall to $50$ percent of its maximum value?
	
	\vspace{14em}
	\item [Exercise 8A.7(a)] ~ (5 points)
	
	The wavefunction of one of the $d$ orbitals is proprotional to $\cos\theta\sin\theta\cos\phi$. At what angles does it have nodal planes?
	
	\vspace{12em}
	\item [Exercise 8A.10(a)] ~ (5 points)
	
	What subshells and orbitals are available in the $M$ shell?

	\vspace{10em}
	\item [Exercise 8A.11(a)] ~ (5 points)
	
	What is the orbital angular momentum (as multiples of $\hbar$) of an electron in the orbitals (1) $1s$, (ii) $3s$, (iii) $3d$? \\Give the numbers of angular and radial nodes in each case.
	
	\vspace{12em}
	\item [Exercise 8B.1(a)] ~ (5 points)
	
	Construct the wavefunction for an excited state of the \ch{He} atom with configuration $1s^12s^1$. Use $Z_{eff}=2$ for the $1s$ electron and $Z_{eff}=1$ for the $2s$ electron.
	
	\vspace{8em}
	\item [Exercise 8B.2(a)] ~ (5 points)
	
	How many electrons can occupy subshells with $l=3$?
	
	\vspace{8em}
	\item [Exercise 8B.4(a)] ~ (5 points)
	
	Write the electronic configuration of the \ch{Ni^{2+}} ion.
	
	\vspace{8em}
	\item [Exercise 8C.3(a)] ~ (5 points)
	
	Which of the following transitions are allowed in the electronic emission specturm of a hydrogenic atom: \\(i) $2s\rightarrow1s$, (ii) $2p\rightarrow1s$, (iii) $3d\rightarrow2p$
	
	
	\vspace{8em}
	\item [Exercise 8C9(a)] ~ (5 points)
	
	What are the possible values of the total spin quantum numbers $S$ and $M_S$ for the \ch{Ni^{2+}} ion?
	
	\vspace{10em}
	\item [Exercise 8C.10(a)] ~ (5 points)
	
	*skip the first part* Which atomic term is likely to lie lowest in energy for the configuration $ns^1nd^1$?
	
	\vspace{12em}
	\item [Exercise 8C.14(a)] ~ (5 points)
	
	Which of the following transitions between terms are allowed in the electronic emission spectrum of a many-electron atom: (i) $^3D_2\rightarrow~^3P_1$ (ii) $^3P_2\rightarrow~^1S_0$, (iii) $^3F_4\rightarrow~^3D_3$?
\end{description}
\end{document}
	