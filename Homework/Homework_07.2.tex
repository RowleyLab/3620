\documentclass[10pt, letterpaper]{memoir}
\usepackage{HomeworkStyle}

\begin{document}
\begin{center}
	{\large Homework 7.2 -- Quantum Models of Motion}
\end{center}

Name: \rule[-.1mm]{15em}{0.1pt}

\begin{description}
	\item [Excercise 7D.1(b)] ~ (5 points)
	
	Evaluate the linear momentum and kinetic energy of a free proton described by the wavefunction $e^{-ikx}$ with $k=5~nm^{-1}$
	
	\vspace{5em}
	\item [Exercise 7D.8(a)] ~ (5 points)
	
	An electron is confined to a square well of length $L$. What would be the length of the box such that the zero-point energy of the electron is equal to its rest mass energy, $m_ec^2$? Express your answer in terms of the parameter $\lambda_C=\nicefrac{h}{m_ec}$, the ``Compton wavelength'' of the electron.
	
	\vspace{5em}
	\item [Exercise 7D.9(a)] ~ (5 points)
	
	For a particle in a box of length $L$  and in the state with $n=3$, at what positions is the probability density at a maximum? At what positions is the probability density zero?
	
	\vspace{5em}	
	\item [Exercise 7D.15(b)] ~ (5 points)
	
	Suppose that a proton of an acidic hydrogen atom is confined to an acid that can be represented by a barrier of height $2.0~eV$ and length $100~pm$. Calculate the probability that a proton with energy $1.4~eV$ can escape from the acid.

	\vspace{5em}	
	\item [Exercise 7E.2(a)] ~ (5 points)
	
	For a certain harmonic oscillator of effective mass $1.33\times10^{-25}~kg$, the difference in adjacent energy levels is $4.82~zJ$. Calculate the force constant of the oscillator
	
	\vspace{5em}
	\item [Exercise 7E.5(a)] ~(5 points)
	
	Assuming the vibrations of a \ch{^{35}Cl2} molecule are equivalent to those of a harmonic oscillator with a force constant $k_f=329~Nm^{-1}$, what is the zero-point energy of vibration of this molecule? Use $m(\ch{^{35}Cl})=34.9688~m_u$.
	
	\vspace{5em}
	\item [Exercise 7E.7(a)] ~ (5 points)
	
	How many nodes are there in the wavefunction of a harmonic oscillator with (i) $v=3$; (ii) $v=4$?
	
	\vspace{5em}
	\item [Exercise 7F.1(b)] ~ (5 points)
	
	The rotation of a molecule an be represented by the motion of a particle moving over the surface of a sphere with angular momentum quantum number $l=2$. Calculate the magnitude of its angular momentum and the possible components of the angular momentum along the z-axis. Express your results as multiples of $\hbar$.
	
	\vspace{5em}
	\item [Exercise 7F.10(a)] ~ (5 points)
	
	How many angular nodes are there for the spherical harmonic $Y_{3,0}$ and at which values of $\theta$ do they occur?
	
	\vspace{5em}
	\item [Exercise 7F.12(a)] ~ (5 points)
	
	What is the degeneracy of a molecule rotating with $J=3$?
	
\end{description}
\end{document}
	