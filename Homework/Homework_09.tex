\documentclass[10pt, letterpaper]{memoir}
\usepackage{HomeworkStyle}

\begin{document}
\begin{center}
	{\large Homework 9 -- Molecular Structure}
\end{center}

Name: \rule[-.1mm]{15em}{0.1pt}

\begin{description}
	\item [Excercise 9A.2(a)] ~ (5 points)
	
	Write the valence-bond wavefunction for the resonance hybrid \ch{HF <-> H^+F^- <-> H^-F^+} (allow for different contributions of each structure)*(For this problem you may neglect the Pauli principle, which is always followed by taking a simple wavefunction and making it more complicated in a rather banal way)*
	
	
	\vspace{9em}
	\item [Excercise 9A.4(a)] ~ (5 points)
	
	Account for the ability of Nitrogen to form four bonds, as in \ch{NH4+}
	
	\vspace{9em}
	\item [Excercise 9A.5(a)] ~ (5 points)
	
	Describe the bonding in 1,3-butadiene using hybrid orbitals
	
	\vspace{9em}
	\item [Excercise 9A.7(a)] ~ (5 points)
	
	Show that the linear combinations $h_1=s+p_x+p_y+p_z$ and $h_2=s-p_x-p_y+p_z$ are mutually orthogonal
	
	\vspace{15em}
	\item [Excercise 9B.4(a)] ~ (5 points)
	
	Identify the $g$ or $u$ character of bonding and antibonding $\pi$ orbitals formed by side-by-side overlap of $p$ atomic orbitals
	
	\vspace{10em}
	\item [Excercise 9C.1(a)] ~ (5 points)
	
	Give the ground-state electron configurations and bond orders of (i) \ch{Li2}, (ii) \ch{Be2}, and (iii) \ch{C2}.
	
	\vspace{22em}
	\item [Excercise 9C.3(a)] ~ (5 points)
	
	Which has the higher dissociation energy, \ch{F2} or \ch{F2^+}?
	
	\vspace{22em}
	\item [Excercise 9D.1(a)] ~ (5 points)
	
	Give the ground-state electron configurations of (i) \ch{CO}, (ii) \ch{NO}, and (iii) \ch{CN^-}
	
	\vspace{25em}
	\item [Excercise 9D.5(a)] ~ (5 points)
	
	Estimate the orbital energies to use in a calculation of the molecular orbitals of HCl. For data, see Tables 8B.4 and 8B.5 (The full tables, in the appendix). Take $\beta=-1.00eV$.
	
	\vspace{13em}
	\item [Excercise 9D.6(a)] ~ (5 points)
	
	Use the values derived in Exercise 9D.5(a) to estimate the molecular orbital energies in \ch{HCl}; use $S=0$
	
	
	\vspace{15em}
	\item [Excercise 9E.1(a)] ~ (5 points)
	
	Set up the secular determinants for (i) linear \ch{H3}, (ii) cyclic \ch{H3} within the H\"uckel approximation.
	
	\vspace{15em}
	\item [Excercise 9E.3(a)] ~ (5 points)
	
	What is the delocalization energy and $\pi$-bond formation energy of (i) the benzene anion, (ii) the benzene cation?
\end{description}
\end{document}
	