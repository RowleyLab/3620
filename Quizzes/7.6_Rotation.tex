\documentclass[11pt, letterpaper]{memoir}
\usepackage{HomeworkStyle}
\geometry{margin=0.75in}



\begin{document}

\begin{center}
	{\large Quiz 7.6 --	Rotational Motion}
\end{center}
{\large Name: \rule[-1mm]{4in}{.1pt}

\subsection*{New Coordinate Systems}
For rotations (and other systems, later) we will use non-cartesian coordinate systems. For cylindrical and spherical polar coordinates give:
\begin{itemize}
	\item The Laplacian operator ($\nabla^2$)
	      \begin{itemize}
		      \item Cylindrical:

		            \vspace{2em}
		      \item Spherical Polar:
	      \end{itemize}

	      \vspace{2em}
	\item The Jacobian (infinitesimal volume element)
	      \begin{itemize}
		      \item Cylindrical:

		            \vspace{2em}
		      \item Spherical Polar:
	      \end{itemize}

	      \vspace{2em}
	\item An integral of function $F(\tau)$ over all space, with the correct limits of integration and Jacobian
	      \begin{itemize}
		      \item Cylindrical:

		            \vspace{2em}
		      \item Spherical Polar:
	      \end{itemize}
\end{itemize}

\vspace{2em}

\subsection*{Rotation and Quantum Numbers}
Quantum mechanical states are labeled by their \emph{quantum numbers}. Give the symbol, name, and relation to observable properties for the quantum numbers in the following systems:
\begin{itemize}
	\item Particle on a Ring

	      \vspace{4em}
	\item Rigid Rotor
\end{itemize}

\subsection*{Rigid Rotor}
Consider a 3-dimensional rigid rotor with a moment of inertia $I=7.4\times10^{-47}~kgm^2$

\begin{itemize}
	\item Give the energy (in $J$) and total angular momentum of the $l=2$ energy level

	      \vspace{15em}
	\item List all of  the allowed values for the $z$-component of the angular momentum

	      \vspace{15em}
	\item List all the observables of a rigid rotor which we can know simultaneously

	      \vspace{6em}
	\item List all pairs of observables for which there exists an uncertainty relationship
\end{itemize}

\newpage
\pagestyle{empty}
\addtocounter{page}{-1}
\newgeometry{margin=1.25in}
\section*{\emph{Holy Sonnets: Death, be not proud}}
\paragraph{By John Donne}~
\begin{verse}
	Death, be not proud, though some have called thee\\
	Mighty and dreadful, for thou art not so;\\
	For those whom thou think'st thou dost overthrow\\
	Die not, poor Death, nor yet canst thou kill me.\\
	From rest and sleep, which but thy pictures be,\\
	Much pleasure; then from thee much more must flow,\\
	And soonest our best men with thee do go,\\
	Rest of their bones, and soul's delivery.\\
	Thou art slave to fate, chance, kings, and desperate men,\\
	And dost with poison, war, and sickness dwell,\\
	And poppy or charms can make us sleep as well\\
	And better than thy stroke; why swell'st thou then?\\
	One short sleep past, we wake eternally\\
	And death shall be no more; Death, thou shalt die.
\end{verse}
\end{document}
