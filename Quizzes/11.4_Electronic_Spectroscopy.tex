\documentclass[11pt, letterpaper]{memoir}
\usepackage{HomeworkStyle}
\geometry{margin=0.75in}



\begin{document}

	\begin{center}
		{\large Quiz 11.4 -- Electronic Spectroscopy}
	\end{center}
	{\large Name: \rule[-1mm]{4in}{.1pt} 

\subsection*{Electronic Term Symbols}
Give the term symbol for the excited state of \ch{C2} with the following electronic configuration: 

\noindent
$\sigma_g(1s)^2\sigma_u^\star(1s)^2\sigma_g(2s)^2\sigma_u^\star(2s)^2\pi_u(2p)^3\sigma_g(2p)^1$


\vspace{2em}\noindent
List all selection rules for electronic transitions

\vspace{3em}
\subsection*{Franck Condon Factors}
An electronic excitation significantly weakens and lengthens a chemical bond. Which vibrational state of the excited electronic state is likely to show the strongest transition? (Generally. I'm not looking for a particular value of $v^\prime$)

\vspace{3em}\noindent
The absorption and fluorescence spectra below show a few vibronic transitions. Give each peak a label indicating the initial and final vibrational states involved in each transition. Vibrational states of the ground electronic state should be referenced by their $v$ quantum number, and vibrational states of the excited electronic state should be referenced by their $v^\prime$ quantum number

\noindent
\includegraphics[width=0.5\linewidth]{Fluorescence}

\noindent
Next to the spectrum above, roughly sketch the potential wells and vibrational states for the electronic states involved.

\newpage
\subsection*{Decay Pathways}
Classify each decay pathway as \emph{internal conversion}, \emph{fluorescence}, \emph{phosphorescence}, or \emph{inter-system crossing}

\begin{itemize}
	\item $S_1 \rightarrow T_1$ (radiationless)
	
	\vspace{1em}
	\item $S_1 \rightarrow S_0$ (radiative)
	
	\vspace{1em}
	\item $S_1 \rightarrow S_0$ (radiationless)
	
	\vspace{1em}
	\item $T_1 \rightarrow S_0$ (radiative)
	
	\vspace{1em}
	\item $T_1,v^\prime=6 \rightarrow T_1,v^\prime=0$ (radiationless)
\end{itemize}
\end{document}
