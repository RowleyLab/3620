\documentclass[11pt, letterpaper]{memoir}
\usepackage{HomeworkStyle}
\geometry{margin=0.75in}



\begin{document}

	\begin{center}
		{\large Quiz 8.1 --	Hydrogenic Atoms}
	\end{center}
	{\large Name: \rule[-1mm]{4in}{.1pt} 

\subsection*{Hydrogen Atomic Emission}

Below are the four visible light hydrogen atomic emission line wavelengths. Convert them into energy units of wavenumbers ($cm^{-1}$), and identify each transition's starting and ending states

\begin{tabular}{c|c|c|c|c}
	Wavelength ($nm$) & $656$ & $486$ & $434$ & $410$ \\ \midrule
	Energy ($cm^{-1}$) & ~\hspace{5em}~& ~\hspace{5em}~&~\hspace{5em}~ &~\hspace{5em}~ \\ \midrule
	States & ~\hspace{5em}~& ~\hspace{5em}~&~\hspace{5em}~ &~\hspace{5em}~
\end{tabular}

\noindent
Consider the same transitions in a hydrogenic C ion (\ch{C^{5+}}). Give the transition energies and wavelengths for the same transitions in this ion.

\begin{tabular}{c|c|c|c|c}
	Energy ($cm^{-1}$) & &  & & \\ \midrule
	Wavelength ($nm$) & ~\hspace{5em}~& ~\hspace{5em}~&~\hspace{5em}~ &~\hspace{5em}~
\end{tabular}

\vspace{2em}
\subsection*{Atomic Orbitals}

Give the number of angular and radial nodes for each of the following atomic orbitals:

{\Large $3s$ \hspace{4em} $3d$ \hspace{4em} $4p$ \hspace{4em} $6f$ \hspace{4em} $6s$ \hspace{4em} $3p$}

\vspace{2em}\noindent
The radial node of a $2s$ wavefunction splits the orbital into two parts, like an onion with only two layers.

\noindent
For a hydrogen atom, the $2s$ atomic orbital is: $\Psi = N_{2s}\left(2-\frac{r}{a_0}\right)e^{-\nicefrac{r}{2a_0}}$

\noindent
Give the radial distance to the first radial node (you may express your answer in terms of the Bohr radius)

\vspace{4em}\noindent
Suppose you wanted to compare the probabilities of finding a $2s$ electron inside or outside of the radial node. Give the integrals you would evaluate to find those probabilities (you don't have to solve them, but if your curiosity grips you it shouldn't be too difficult to do so)

\end{document}
