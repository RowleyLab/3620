\documentclass[11pt, letterpaper]{memoir}
\usepackage{HomeworkStyle}
\geometry{margin=0.75in}



\begin{document}

\begin{center}
	{\large Quiz 8.2 -- Multi-Electron Atoms}
\end{center}
{\large Name: \rule[-1mm]{4in}{.1pt}

\section*{Electronic Configurations}
List the three principles which lead to proper electronic configurations

\vspace{5em}\noindent
Many of the elements break from the normal pattern for electronic configurations. There are three primary modes for these deviations. Give an example of an element, with its configuration, for each of these modes

\vspace{7em}\noindent
What do these exceptions to the normal pattern tell us qualitatively about the orbital energies?

\vspace{7em}\noindent
\section*{Spin States of Multi-Electron Atoms}
Consider the excited He electronic state with a configuration $1s^1~2s^1$. This configuration can give rise to both singlet and triplet terms

\noindent Draw energy level diagrams which illustrate the difference between these excited states

\vspace{7em} \noindent
Explain one experimental difference between the singlet and triplet states of excited \ch{He}

\vspace{7em}
\section*{The Pauli Principle}
For the singlet state of excited \ch{He}, give a valid total wavefunction which obeys the Pauli Principle

\vspace{7em} \noindent
For the triplet state of excited \ch{He}, give two valid total wavefunctions which obey the Pauli Principle (A third one exists, but it involves a new paradigm to derive it so we will leave it alone for now)

\vspace{10em}
\section*{True Multi-Electron Wavefunctions}
True electron orbitals for multi-electron atoms are not actually identical to the hydrogenic orbitals. What theory is used to approximate the true wavefunctions for multi-electron atoms, and what factor limits its accuracy?

\newpage
\pagestyle{empty}
\addtocounter{page}{-1}
\newgeometry{margin=1.25in}
\section*{\emph{A Litany for Survival}}
\paragraph{By Audre Lorde}~

\vspace{1em}\noindent
\begin{minipage}[t]{0.56\linewidth}
	For those of us who live at the shoreline\\
	standing upon the constant edges of decision\\
	crucial and alone\\
	for those of us who cannot indulge\\
	the passing dreams of choice\\
	who love in doorways coming and going\\
	in the hours between dawns\\
	looking inward and outward\\
	at once before and after\\
	seeking a now that can breed\\
	futures\\
	like bread in our children’s mouths\\
	so their dreams will not reflect\\
	the death of ours;

	For those of us\\
	who were imprinted with fear\\
	like a faint line in the center of our foreheads\\
	learning to be afraid with our mother’s milk\\
	for by this weapon\\
	this illusion of some safety to be found\\
	the heavy-footed hoped to silence us\\
	For all of us\\
	this instant and this triumph\\
	We were never meant to survive.
\end{minipage}
\begin{minipage}[t]{0.45\linewidth}
	And when the sun rises we are afraid\\
	it might not remain\\
	when the sun sets we are afraid\\
	it might not rise in the morning\\
	when our stomachs are full we are afraid\\
	of indigestion\\
	when our stomachs are empty we are afraid\\
	we may never eat again\\
	when we are loved we are afraid\\
	love will vanish\\
	when we are alone we are afraid\\
	love will never return\\
	and when we speak we are afraid\\
	our words will not be heard\\
	nor welcomed\\
	but when we are silent\\
	we are still afraid

	So it is better to speak\\
	remembering\\
	we were never meant to survive.
\end{minipage}
\end{document}
