\documentclass[11pt, letterpaper]{memoir}
\usepackage{HomeworkStyle}
\geometry{margin=0.75in}



\begin{document}

	\begin{center}
		{\large Quiz 8.2 -- Multi-Electron Atoms}
	\end{center}
	{\large Name: \rule[-1mm]{4in}{.1pt} 

	Consider the excited He electronic state with a configuration $1s^1~2p^1$. This configuration can give rise to both singlet and triplet terms

\noindent $\circ$ Explain two differences between the singlet and triplet states, and tell which one can be reached from the ground state of He through an allowed transition



\vspace{20em}
\noindent $\circ$ For the singlet state, give a valid total wavefunction which obeys the Pauli Principle

\newpage
\section*{Problem 5 (10 points)}
Give the lowest energy term for each of the following electronic configurations:

\begin{description}
	\item[\ch{Cl:[Ne]}$3s^2~3p^5$]~
	
	\vspace{4em}
	\item[\ch{C:[He]}$2s^2~2p^2$] ~
	
	\vspace{4em}
	\item[\ch{Ti:[Ar]}$4s^2~3d^2$] ~
	
	\vspace{4em}
	\item[\ch{Si^*:[Ne]}$3s^1~3p^2~4p^1$] ~
	
	\vspace{4em}
	\item[\ch{Nd:[Xe]}$6s^2~4f^4$]
\end{description}
\end{document}
