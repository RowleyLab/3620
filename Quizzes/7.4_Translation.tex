\documentclass[11pt, letterpaper]{memoir}
\usepackage{HomeworkStyle}
\geometry{margin=0.75in}



\begin{document}

	\begin{center}
		{\large Quiz 7.4 --	Translational Motion}
	\end{center}
	{\large Name: \rule[-1mm]{4in}{.1pt} 

\subsection*{Particle in a Box}
Consider a particle-in-a box system where the particle is confined between $x=0$ and $x=1$ (i.e. $L=1$). If the system is in a superposition state where $\Psi(x) = \dfrac{1}{\sqrt{2}}\psi_1(x) + \dfrac{1}{\sqrt{2}}\psi_2(x)$, then what is the average position $\avg{\hat{x}}$ for that state? (the x-position operator is simply multiplying the function by $x$, i.e.  $\hat{x}=x$)


\vspace{46em}\noindent
*This is actually a very interesting problem when time-dependence is considered. We won't explore it in this class, but if we included the time dependence we would see the probability function bouncing back and forth in the box. That motion has been animated in figure 3 of the article found at: http://en.citizendium.org/wiki/Particle\_in\_a\_box*

\subsection*{Quantum Well}
Consider a particle confined to a 2-dimensional box. This system is commonly called a \emph{quantum well}. If the two sides are equal in length, give the energies and degeneracies to the first six energy levels of this quantum system


\subsection*{Tunneling}
Consider an electron approaching a potential energy barrier. The barrier is $5.0~nm$ wide, and $6.0\times 10^{-23}~J$ high, while the electron has kinetic energy of $1.0\times 10^{-23}~J$

\noindent What will be the probability that the electron is transmitted through the barrier?

\end{document}
