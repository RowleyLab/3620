\documentclass[11pt, letterpaper]{memoir}
\usepackage{HomeworkStyle}
\geometry{margin=0.75in}



\begin{document}

	\begin{center}
		{\large Quiz 7.4 --	Translational Motion}
	\end{center}
	{\large Name: \rule[-1mm]{4in}{.1pt} 

\subsection*{Particle in a Box}
Consider a H atom confined in a box with $L=1.5~nm$. Model this system as a particle in a box. For each of the first three energy levels, draw the wavefunction and give the energy. Point out any nodes on your drawn wavefunctions.


\vspace{10em}\noindent
Find an expression for the spacing between energy levels for a particle in a box ($E_{n+1}-E_n$), and describe its trend, if any.

\vspace{6em}
\subsection*{Quantum Well}
Consider a particle confined to a 2-dimensional box. This system is commonly called a \emph{quantum well}. If the two sides are equal in length, give the energies and degeneracies to the first four energy levels of this quantum system

\vspace{20em}
\subsection*{Tunneling}
Consider an electron approaching a potential energy barrier. The barrier is $5.0~nm$ wide, and $4.0\times 10^{-23}~J$ high, while the electron has kinetic energy of $1.0\times 10^{-23}~J$

\noindent What will be the probability that the electron is transmitted through the barrier?

\vspace{15em}\noindent
Sketch this system below, showing both the potential energy and the electron wavefunction in qualitative terms.
\end{document}
