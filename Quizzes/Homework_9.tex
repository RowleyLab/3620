\documentclass[10pt, letterpaper]{memoir}
\usepackage{HomeworkStyle}

\begin{document}
\begin{center}
	{\large Homework 9 -- Atomic Structure and Spectra}
	
	Due: Friday, February 16 \hspace{3em} Points: ${\dfrac{~}{~~60~~}}$
\end{center}

Name: \rule[-.1mm]{15em}{0.1pt}	
	
	\section*{Problem 1  (10 points)}
	There are four spectral lines in the hydrogen atom which are within the visible light spectrum. Their wavelengths are given below:
	
	\begin{tabular}{c|c|c|c|c}
		Color & Red & Aqua & Blue & Violet\\ \midrule
		Wavelength (nm) & $656$ & $486$ & $434$ & $410$
	\end{tabular}
	
	\noindent $\circ$ Give the starting and ending states for each of these transitions
	
	\vspace{18em}
	
	\noindent $\circ$ A \ch{He^{+}} ion will show similar spectral lines, with an energy shift due to the greater nuclear charge. Find the wavelengths for the same transitions in a \ch{He^{+}} ion (remember to find $\mu$ and $\tilde{R}_N$)
	
	\newpage	
	\section*{Problem 2 (10 points)}
	Give the $n$ and $l$ quantum numbers for the following sketched hydrogenic orbitals
	

	\vspace{20em}
	\section*{Problem 3 (10 points)}
	Calculate the most probable radius for a H atom 2s orbital, and for a \ch{He^{+}} ion 2s orbital. Explain why the most probable radii are different, even though they are the same orbital type
	
	\newpage
	\section*{Problem 4 (10 points)}
	Consider the excited He electronic state with a configuration $1s^1~2p^1$. This configuration can give rise to both singlet and triplet terms
	
	\noindent $\circ$ Explain two differences between the singlet and triplet states, and tell which one can be reached from the ground state of He through an allowed transition
	
	
	
	\vspace{20em}
	\noindent $\circ$ For the singlet state, give a valid total wavefunction which obeys the Pauli Principle
	
	\newpage
	\section*{Problem 5 (10 points)}
	Give the lowest energy term for each of the following electronic configurations:
	
	\begin{description}
		\item[\ch{Cl:[Ne]}$3s^2~3p^5$]~
		
		\vspace{4em}
		\item[\ch{C:[He]}$2s^2~2p^2$] ~
		
		\vspace{4em}
		\item[\ch{Ti:[Ar]}$4s^2~3d^2$] ~
		
		\vspace{4em}
		\item[\ch{Si^*:[Ne]}$3s^1~3p^2~4p^1$] ~
				
		\vspace{4em}
		\item[\ch{Nd:[Xe]}$6s^2~4f^4$]
	\end{description}
	
	\vspace{3em}
	\section*{Problem 6 (10 points)}
	Tell whether each transition is allowed. If not, give the selection rule which it violates
	
	\begin{itemize}
		\item $1s^1\rightarrow2s^1$
		
		\vspace{4em}
		\item $1s^2~2s^2~2p^2\rightarrow 1s^1~2s^2~2p^3$
		
		\vspace{4em}
		\item $^3P_2 \rightarrow ~ ^3S_1$
		
		\vspace{4em}
		\item $^1D_2 \rightarrow ~ ^3P_2$
		
		\vspace{4em}
		\item $^3D_1 \rightarrow ~ ^3S_1$
	\end{itemize}

\end{document}
	