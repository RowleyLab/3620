\documentclass[11pt, letterpaper]{memoir}
\usepackage{HomeworkStyle}
\geometry{margin=0.75in}



\begin{document}

	\begin{center}
		{\large Quiz 9.2 -- Molecular Orbital Theory: Diatomic Molecules}
	\end{center}
	{\large Name: \rule[-1mm]{4in}{.1pt} 

\subsection*{Homonuclear Diatomics}

\ch{O2^{4+}} and \ch{N2^{2+}} have the same number of electrons, so you might expect them to have identical electronic structure

\noindent $\circ$ Draw the molecular orbital energy level diagram for these two molecules, filled with the proper number of electrons

\vspace{30em}
\noindent $\circ$ Give the bond order of both molecules

\vspace{5em}
\noindent $\circ$ Describe how both molecules might interact with a strong magnetic field

\newpage
\subsection*{Heteronuclear Diatomics}

Consider the molecule HF. Because of the much higher nuclear charge on F, the H$1s$ orbital actually aligns best energetically with the F$2p_z$ orbital, so they are the two which combine to form a molecular orbital. $\alpha_{H1s} = -7.2eV$, $\alpha_{F2p}=-10.4eV$, and $\beta_{H1s-F2p}=-1.0eV$

\noindent $\circ$ Calculate the energies of the two molecular orbitals, and draw an energy-level diagram which includes both the energies of the atomic orbitals and molecular orbitals. Remember that for heteronuclear diatomics we usually assume that the overlap integral $S=0$

\vspace{27em}
\noindent $\circ$ Calculate the coefficients for both MOs and sketch how they might look considering the unequal contributions from both atoms

\end{document}
