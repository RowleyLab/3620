\documentclass[10pt, letterpaper]{memoir}
\usepackage{HomeworkStyle}

\begin{document}
\begin{center}
	{\large Homework 7 -- Introduction to Quantum Theory}
	
	Due: Friday, January 26 \hspace{3em} Points: ${\dfrac{~}{~~60~~}}$
\end{center}

Name: \rule[-.1mm]{15em}{0.1pt}	
	
	\section*{Problem 1  (10 points)}
	You wish to prove the wavelike nature of He atoms. You plan to generate a coherent beam of He ions traveling at $100.0~\nicefrac{m}{s}$ and direct them toward a grating to observe diffraction. What is the characteristic wavelength of the He ions in the beam? 

	\vspace{15em}	
	\section*{Problem 2 (10 points)}
	For a particle confined in the region $0\leq x \leq L$, the appropriate wavefunctions are:	
	\begin{equation*}
		\psi_n(x) = \sqrt{\dfrac{2}{L}}\sin{\frac{n\pi x}{L}}
	\end{equation*}
		
	\noindent
	Another function, $X(x) = -4x^2+4x$ has a similar shape and obeys the same boundary conditions. Prove whether or not this function is also a solution to the Schr\"odinger equation.
	
	\vspace{25em}
	\section*{Problem 3 (10 points)}
	The wavefunction for a 1s electron orbital is:	
	\begin{equation*}
		\psi_{1s}(r,\theta,\phi) = e^{-\nicefrac{r}{a_0}}
	\end{equation*}
	
	\noindent
	Note that this is a function in spherical polar coordinates, and that $a_0$ is the Bohr radius. Find the normalization constant, and give the complete normalized wavefunction $\psi_{1s}(r,\theta,\phi)$
	
	\vspace{22em}
	\section*{Problem 4 (10 points)}
	Consider two even functions:
	\begin{equation*}
		\phi_1(x) = \sqrt{\dfrac{3}{2}}x \hspace{4em} \phi_2(x) = \sqrt{\dfrac{175}{8}}\left(x^3-\dfrac{3}{5}x\right)
	\end{equation*}
	
	Show that these two functions are orthogonal over the interval $[-1,1]$
	
	\vspace{22em}
	\section*{Problem 5 (10 points)}
	For electronic orbitals, we can define an orbital angular momentum operator: $\hat{l}^2$
	
	\noindent Some eigenvalues are:
	
	$\hat{l}^2\psi_{3s} = 0$
	
	$\hat{l}^2\psi_{3p} = 2 \hbar^2\psi_{3p}$
	
	$\hat{l}^2\psi_{3d} = 6 \hbar^2\psi_{3d}$
	
	\noindent
	If an electron is in the superposition state $\Psi = \left(\dfrac{1}{\sqrt{2}}\psi_{3s} + \dfrac{1}{\sqrt{3}}\psi_{3p} + \dfrac{1}{\sqrt{6}}\psi_{3d}\right)$, what will be the expectation value $\avg{\hat{l}^2}$?
	
	\vspace{20em}
	\section*{Problem 6 (10 points)}

	Consider two operators: $\hat{A} = \dfrac{\mathrm{d}^2}{\mathrm{d}x\mathrm{d}y}$ and $\hat{B} = x^2 + y$
	
	\noindent Give the commutator $\left[\hat{A},\hat{B}\right]$. You may assume that the operators will only be used with functions that are separable. i.e. You may use a trial function of $f(x)g(y)$
\end{document}
	