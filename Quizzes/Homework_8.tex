\documentclass[10pt, letterpaper]{memoir}
\usepackage{HomeworkStyle}

\begin{document}
\begin{center}
	{\large Homework 8 -- The Quantum Theory of Motion}
	
	Due: Friday, February 2 \hspace{3em} Points: ${\dfrac{~}{~~70~~}}$
\end{center}

Name: \rule[-.1mm]{15em}{0.1pt}	
	
	\section*{Problem 1  (10 points)}
	Consider a particle-in-a box system where the particle is confined between $x=0$ and $x=1$ (i.e. $L=1$). If the system is in a superposition state where $\Psi(x) = \dfrac{1}{\sqrt{2}}\psi_1(x) + \dfrac{1}{\sqrt{2}}\psi_2(x)$, then what is the average position $\avg{\hat{x}}$ for that state? (the x-position operator is simply multiplying the function by $x$, i.e.  $\hat{x}=x$)


	\vspace{46em}
	*This is actually a very interesting problem when time-dependence is considered. We won't explore it in this class, but if we included the time dependence we would see the probability function bouncing back and forth in the box. That motion has been animated in figure 3 of the article found at: http://en.citizendium.org/wiki/Particle\_in\_a\_box*
	\newpage
	\section*{Problem 2 (10 points)}
	Consider an electron approaching a potential energy barrier. The barrier is $5.0~nm$ wide, and $6.0\times 10^{-23}~J$ high, while the electron has kinetic energy of $1.0\times 10^{-23}~J$
	
	\noindent What will be the probability that the electron is transmitted through the barrier?
	
	\vspace{25em}
	\section*{Problem 3 (10 points)}
	Consider a particle confined to a 2-dimensional box. This system is commonly called a \emph{quantum well}. If the two sides are equal in length, give the energies and degeneracies to the first six energy levels of this quantum system
	
	\vspace{22em}
	\section*{Problem 4 (10 points)}
	Consider a quantum mechanical harmonic oscillator with mass equal to $8.5~AMU$ and a force constant of $400\dfrac{N}{m}$. What will be the zero-point energy in ($J$) for this system?


	\vspace{25em}
	\section*{Problem 5 (10 points)}
	Consider the same quantum mechanical harmonic oscillator introduced above in the first excited state ($v=1$). The normalization constant for this state is: $N_1=1.94\times10^6$. You are interested in finding the probability that the oscillator will be found with a displacement between $x=-0.1~pm$ and $x=0.1~pm$.
	
	\noindent Give the integral which you would evaluate to find that probability (including the appropriate limits of integration). Also, sketch the probability distribution function with the integrated area shaded.
	
	\vspace{25em}
	\section*{Problem 6 (20 points)}
	Consider a 3-dimensional rigid rotor with a moment of inertia $I=7.4\times10^{-47}~kgm^2$
	
	\begin{itemize}
		\item Give the energy (in $J$) and total angular momentum of the $l=2$ energy level
		
		\vspace{15em}
		\item List all of  the allowed values for the $z$-component of the angular momentum
	\end{itemize}

\end{document}
	