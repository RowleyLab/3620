\documentclass[10pt, letterpaper]{memoir}
\usepackage{HWStyle}

\begin{document}
	\begin{center}
		{\Huge CHEM 3610}
		{\LARGE-- Fall 2016
		
		Homework 1
		
		Due: September 2, 2016}
	\end{center}
		
	\section*{Problem 1 (20 points)}
	A spectrometer uses a slit at its entrance to improve resolution. When using an array detector (such as a CCD), the slit would optimally be as narrow as the individual detector elements on the array. However, a too-narrow slit will reduce both resolution and sensitivity through diffraction. A good rule of thumb is that the slit width should be at least 100 times the wavelength of light to be measured. 
	
	Your spectrometer has a linear array detector with 2048 pixels and an active area 1 in long. Could you set the slit to an optimal width and measure visible light without having diffraction problems?
	\vspace{10em}
	
	What is the longest wavelength that could be reliably measured with an optimal slit width?
	\vspace{10em}
	
	Consider this longest wavelength and the optimal slit width calculated above. Given that the optical pathlength within the spectrometer is 20 cm, what is the distance to the first dark fringe for this wavelength and slit width?
	\vspace{10em}
	
	Was our rule of thumb reliable for this configuration? If not, which direction should it be adjusted (wider or narrower slit for a given wavelength of light)?
	\vspace{10em}
	
	\section*{Problem 2  (10 points)}
	You wish to prove the wavelike nature of He atoms. You plan to generate a coherent beam of He ions traveling at $10\nicefrac{m}{s}$, and direct them through a slit with a width of $1 \mu m$. To detect the diffraction pattern, you have a microchannel plate array (MCP) which will detect the ions, but the array is relatively small. The fringe spacing ($s$) must be $0.20 cm$. How far behind the slit should you position the MCP.
	% \lambda = 9.9699 nm, d=0.2006 m 
	\vspace{16em}
	
	\section*{Problem 3 (10 points)}
	Figure 1 shows a classical standing wave. 
	\begin{figure}[h]
		\centering
		\includegraphics[width=0.7\linewidth]{../Lecture_Notes/figures/CWave}
		
		Figure 1: A Classical Standing Wave
		\label{fig:cwave}
	\end{figure}
	
	In class we proved the spatial solution:
	\begin{equation*}
	X(x) = B\sin{\frac{n\pi x}{l}}
	\end{equation*}
		
	
	
	Another function, $X(x) = -4x^2+4x$ has similar shape and obeys the same boundary conditions. Prove whether or not this function is another solution to the non-dispersive wave equation for this system.

	
\end{document}
	